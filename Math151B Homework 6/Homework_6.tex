\documentclass[12pt]{article}
\setlength{\oddsidemargin}{0in}
\setlength{\evensidemargin}{0in}
\setlength{\textwidth}{6.5in}
\setlength{\parindent}{0in}
% \setlength{\parskip}{\baselineskip}
\usepackage{graphicx}
\usepackage{multirow}
\usepackage{multicol}
\usepackage{listings}
\usepackage{color}

\definecolor{dkgreen}{rgb}{0,0.6,0}
\definecolor{gray}{rgb}{0.5,0.5,0.5}
\definecolor{mauve}{rgb}{0.58,0,0.82}

\lstset{frame=tb,
  language=MATLAB,
  aboveskip=3mm,
  belowskip=3mm,
  showstringspaces=false,
  columns=flexible,
  basicstyle={\small\ttfamily},
  numbers=none,
  numberstyle=\tiny\color{gray},
  keywordstyle=\color{blue},
  commentstyle=\color{dkgreen},
  stringstyle=\color{mauve},
  breaklines=true,
  breakatwhitespace=true,
  tabsize=3
}

\usepackage{amsmath,amssymb,amsrefs}

\usepackage[top=24mm, bottom=18mm, left=15mm, right=13mm]{geometry}

\usepackage{url}

\begin{document}

\begin{flushright}
Lawrence Ouyang\\
504128219\\
Math 151B\\
\end{flushright}

\begin{center}
Homework 6
\end{center}

S1: Using the Householder reflection code below:
\begin{lstlisting}
function [ A ] = Householder( n, A )
%
% Householder(n, A)
% Obtains symmetric diagonal matrix A(n-1) similar to the symmetric matrix
% A using Householder's method.
% n: dimension
% A: symmetric matrix

for k = 1:(n-2)
    q = 0; alpha = 0; PROD = 0;
    v = zeros(n,1); u = zeros(n,1); z = zeros(n,1);
    
    for j = (k+1):n
        q = q + A(j,k)^2;
    end
    
    if A(k+1,k) == 0
        alpha = -sqrt(q);
    else
        alpha = -(sqrt(q)*A(k+1,k)/abs(A(k+1,k)));
    end
    
    RSQ = alpha^2 - alpha*A(k+1,k);
    v(k+1) = A(k+1,k) - alpha;
    
    for j = (k+2):n
        v(j) = A(j,k);
    end
    
    for j = k:n
        for i = (k+1):n
            u(j) = u(j) + A(j,i)*v(i);
        end
        u(j) = u(j)/RSQ;
    end
    
    for i = (k+1):n
        PROD = PROD + v(i)*u(i);
    end
    
    for j = k:n
        z(j) = u(j) - (PROD/(2*RSQ))*v(j);
    end
    
    for l = (k+1):n-1
        for j = (l+1):n
            A(j,l) = A(j,l) - v(l)*z(j) - v(j)*z(l);
            A(l,j) = A(j,l);
        end
        A(l,l) = A(l,l) - 2*v(l)*z(l);
    end
    A(n,n) = A(n,n) - 2*v(n)*z(n);
    
    for j = (k+2):n
        A(k,j) = 0;
        A(j,k) = 0;
    end
    A(k+1,k) = A(k+1,k) - v(k+1)*z(k);
    A(k,k+1) = A(k+1,k);
end
end
\end{lstlisting}
Applying our formula to the given A results in the following symmetric tridiagonal matrix:
\[
A =
\begin{bmatrix}
4 & 1 & -1 & 0 \\
1 & 3 & -1 & 0 \\
-1 & -1 & 5 & 2 \\
0 & 0 & 2 & 4
\end{bmatrix}
\rightarrow
T =
\begin{bmatrix}
4 & -\sqrt{2} & 0 & 0 \\
-\sqrt{2} & 5 & -\sqrt{3} & 0 \\
0 & -\sqrt{3} & 5 & 0 \\
0 & 0 & 0 & 2 \\
\end{bmatrix}
\]
\\
S2: Using a simple implementation of Rayleigh Quotient Iteration below:
\begin{lstlisting}
function [lambda, v] = RayleighQuotient(n, A, v, N)
%
% RayleighQuotient(n, A, v, N)
% Approximate an eigenvalue and eigenvector of A using the Rayleigh
% Quotient Iteration.
% n: dimension of A
% A: matrix
% v: approximate eigenvector x, norm(x, inf) = 1
% N: maximum number of iterations
% lambda: approximate eigenvalue

lambda = (v.'*A*v)/(v.'*v);
for k = 1:N
    w = (A - lambda*eye(n))\v;
    v = w/norm(w, inf);
    lambda = (v.'*A*v)/(v.'*v);
end
end
\end{lstlisting}
This gives us
\[
\lambda = 2.4859, v =
\begin{bmatrix}
-1.5156\\
1.6038\\
-0.6910\\
0.9127\\
\end{bmatrix}
\]
\\
S3: Using the simple implementation of Simulataneous Iteration below:
\begin{lstlisting}
function [lambda, W] = SimultaneousIteration(A, V, N)
%
% SimultaneousIteration(A, V, N)
% Approximate an eigenvalue and eigenvector of A using the Simultaneous
% Iteration method.
% A: matrix
% V: matrix of orthogonal column entries
% N: maximum number of iterations
% lambda: approximate eigenvalue vector
% W: matrix with eigenvector columns

[Q,R] = qr(V);

for k = 1:N
    W = A*Q;
    [Q,R] = qr(W);
end
lambda = diag(Q.'*A*Q);
end
\end{lstlisting}
The two largest eigenvalues are:
\[
\lambda_1 = 7.0861, v_1 =
\begin{bmatrix}
-2.3562\\-1.8929\\5.3786\\3.4856
\end{bmatrix}
\]
\[
\lambda_2 = 4.4280, v_2 = 
\begin{bmatrix}
-3.1861\\-1.8739\\-0.5102\\-2.3841
\end{bmatrix}
\]
\end{document}